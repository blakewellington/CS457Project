% Author: Robert Konell
% Author: Blake Wellington

\documentclass{article}

\usepackage{fancyhdr}
\usepackage{amssymb}
\usepackage{amsmath}
\usepackage{color}
\usepackage{graphicx}
\usepackage{bm}
%\usepackage{tikz-er2}
%\usepackage{enumitem}
\usepackage{enumerate}
%\usepackage[LGRgreek]{mathastext}
%\usepackage{txfonts}
%\usepackage{siunitx}
%\usepackage{algorithm}
%\usepackage{algpseudocode}
\usepackage{tikz}
\usepackage{ifthen}
%\usepackage{xstring}
%\usepackage{pgfopts}
%\usepackage{tikz-qtree}
\usepackage{calc}
%\usepackage{tikz-uml}
%\usepackage{ulem}
%\usepackage{listings}
%\usepackage{pstricks,pst-uml,pstricks-add}
%\usepackage{pgffor}
%\usepackage{pgfplots}
%\usepgfplotslibrary{polar}
%\usepgflibrary{shapes.geometric}
%\usepackage{fp}
\usetikzlibrary{positioning}
\usetikzlibrary{shadows}
%\usetikzlibrary{arrows,automata,positioning,calc}
%\tikzstyle{vertex}=[draw,fill=black!15,circle,minimum size=18pt,inner sep=0pt]
%\pgfplotsset{my style/.append style={axis x line=middle, axis y line=middle, xlabel={$x$}, ylabel={$y$}, axis equal }}
%\renewcommand{\algorithmiccomment}[1]{\hskip2em$\triangleright$ #1}



%\newcommand{\sim}{\textasciitilde }
\newcommand{\conj}{\;\&\;}
\newcommand{\disj}{\; \vee \;}
\newcommand{\ntilde}{_{\widetilde{~}}}
\newcommand{\dntilde}{_{\widetilde{~}\, \widetilde{~}}}
\newcommand{\ob}{\\ {\bfseries O}}
\newcommand{\cb}{\\ {\bfseries X}}
\newcommand{\subspace}{\hspace*{4em}}


\tikzstyle{every entity} = [top color=white, bottom color=blue!30, 
                            draw=blue!50!black!100, drop shadow]
\tikzstyle{every weak entity} = [drop shadow={shadow xshift=.7ex, 
                                 shadow yshift=-.7ex}]
\tikzstyle{every attribute} = [top color=white, bottom color=yellow!20, 
                               draw=yellow, node distance=1cm, drop shadow]
\tikzstyle{every relationship} = [top color=white, bottom color=red!20, 
                                  draw=red!50!black!100, drop shadow]
\tikzstyle{every weak relationship} = [drop shadow={shadow xshift=.7ex, 
																				shadow yshift=-.7ex}]
\tikzstyle{every isa} = [top color=white, bottom color=green!20, 
                         draw=green!50!black!100, drop shadow]

\usepackage[letterpaper,
            left=3cm,right=3cm,
            top=3cm,bottom=3cm]{geometry}

\lhead{CS457}
\rhead{Project Proposal}
\chead{}                                % central header
\cfoot{}                                   % central footer (cleared)
\rfoot{Konell,Wellington \itshape \thepage}                       % right footer

\title{Project Proposal}
\author{Robert Konell, Blake Wellington}
\date{\today}

\begin{document}
\maketitle
\thispagestyle{fancy}
\pagestyle{fancy}

\section*{Summary}
The application will use data from Tri-Met's database. (Tri-Met is a 
the public transit system in Portland, Oregon). The database contains
route information for all of Tri-Met's trains and buses.

This proposal implies a joint project to be developed by Robert Konell and
Blake Wellington. We recognize that the intent of the CS457 project is to
have student work individually. However, we believe that, by working together,
we can accomplish a greater level of sophistication in the end product.
Each student has skills and interests that can be leveraged to complete
a more robust application with a richer user experience.

Since we are new to Haskell and have limited time to work on this project,
we have broken the project down into manageable, intermediate goals.

\section*{Project Goals}
The aim of this project is to use the Haskell programming language to 
create a web-based application. It is intended as a learning exercise
with the following project goals:

\begin{itemize}
\item Learn functional programming with Haskell
\item Learn how to use a REST API as a web service
\item Learn how to use Haskell libraries to present the data in a meaningful way
\end{itemize}
  
\subsection*{Initial Goal}
The initial goal of the project is to successfully pull a single, small
piece of information from the database, massage it in Haskell, and present
it on a web page. This may be something as basic as showing the next bus/train
to arrive at a given stop number.

\subsection*{Secondary Goal}
Once the initial goal is in place, we will expand the user interface to contain
ever more complex query abilities.  For example, we might add the ability to 
select a certain route, list the stops of a route, estimate the travel time
between two stops, plan a trip (including multiple transfers), etc.

\subsection*{Final Goal}
Since this is a web-based application, it would be nice to have it hosted on
a public server. The final goal is to be able to pack all the code up and 
install it on a server in the cloud (such as Amazon EC2). This will involve
installing Haskell (or just shipping the compiled version) on a server 
equipped with Apache web server.

\section*{Project Motivation} 
The authors of this application are Computer Science students at Portland
State University. We are relatively new to programming and would like to gain
experience in building applications of greater complexity than those developed
for small academic projects. As part of our requirement to use functional
programming, we will use Haskell and associated libraries to create an integrated 
piece of software with 'real world'-like behavior.

This includes using graphical libraries and/or HTML wrapper libraries for the 
front end, and calls to an exposed web API for the backend. The final user
interface will be presented on a web page using HTML (and possibly some Javascript).

\section*{Collaboration Plan}
A git repository hosted on Github (https://github.com/blakewellington/cs457project) 
will be used.  One benefit of using Github is that it will provide a public way
for each student's work to be individually tracked.  

In the first week a high-level architecture will be designed, with modules and 
their sub-tasks identified and broken down into manageable chunks.  Each student
will take ownership of certain tasks, but not entire modules, that way both 
students will have a hand in every part of the project.  

Occasional pair programming will be utilized as time allows.

\section*{Schedule:}

\subsubsection*{Week of Feb 17th:}
\begin{itemize}
\item High level design, discussion of major modules, identification of bite sized tasks.
\item Research and analyze Trimet web API.
\item Research and analyze web server options.
\end{itemize}

\subsubsection*{Week of Feb 24th:}
\begin{itemize}
\item Functions to call Trimet API created.
\item Work begins to parse responses, including design of data types to hold parsed data.
\item Web server option chosen, work begins on having server accept GET and response
with Haskell generated minimal HTTP page.
\end{itemize}

\subsubsection*{Week of March 3rd:}
\begin{itemize}
\item Parser is working.
\item A complete cycle of: HTTP Get, haskell calls API, parse response, generate page and respond
\item For at least 1 API function.  
\item Work begins on designing the front end more with a complete cycle complete.
\end{itemize}

\subsubsection*{Week of March 10th:}
\begin{itemize}
\item More functions created for more complex queries.
\item Heavy work on the front end/HTML in order to make it robust enough to use backend.
\item Begin cleaning up code by end of week.
\item Work begins on final paper.
\end{itemize}

\subsubsection*{Week of March 17th:}
\begin{itemize}
\item Final touches are made.  Paper turned in.  Celebration ensues.
\end{itemize}

\end{document} 
